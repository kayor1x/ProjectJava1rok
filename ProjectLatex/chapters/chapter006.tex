\chapter{Podsumowanie}
\label{cha:podsumowanie}

Ten projekt jest moim pierwszym projektem w ogóle, zdecydowałem się zaimplementować go w JavaFX i Hibernate, z którymi jestem mniej zaznajomiony, ponieważ w laboratoriach używaliśmy JDBC i Swing, które moim zdaniem są łatwiejsze. Ale tak czy inaczej, do tej pory zaimplementowałem następujące funkcje:
\begin{itemize}
    \item Wygodny i odpowiadający tematowi DOTA 2 interfejs użytkownika.
    \item Rejestracja, logowanie użytkowników.
    \item Zarządzanie poziomami dostępu.
    \item Haszowanie haseł użytkowników w bazie danych.
    \item Dodawanie, edytowanie i usuwanie turniejów (wszystko jest połączone z db).
    \item Przeglądanie listy turniejów oraz szczegółowych informacji o każdym z nich.
    \item Wybór typu turnieju (jeżeli offline - trzeba wprowadzić lokalizację) i organizatora turnieju z listy dostępnych (tylko dla administratora).
\end{itemize}

W przyszłości planuję dodać więcej funkcji, takich jak:
\begin{itemize}
    \item Po wyborze turnieju, zwiększyć funkcjonalności dla użytkownika widzieć informacje o drużynach, graczach, meczach tego turnieju.
    \item Dla administratora możliwość zarządzania wszystkimi aspektami turnieju.
    \item Dodać możliwość przeglądania nowości o turniejach DOTA 2(web integracja).
    \item Ulepszenie już zrobionych funkcji.
\end{itemize}